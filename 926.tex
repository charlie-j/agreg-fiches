
\documentclass{agregfiche}

\title{Leçon 926 - Analyse des algorithmes : complexité. Exemples.}

\begin{document}
\maketitle

\secrapports
\begin{rapport}{2017,2018,2019}
l s’agit ici d’une leçon d’exemples. Le candidat doit prendre soin de proposer l’analyse d’algo-rithmes portant sur des domaines variés, avec des méthodes d’analyse également variées : approche combinatoire ou probabiliste, analyse en moyenne ou dans le pire cas. Si la complexité en temps est centrale dans la leçon, la complexité en espace ne doit pas être négligée. La notion de complexité amortie a également toute sa place dans cette leçon, sur un exemple bien choisi, commeunion find(ce n’est qu’un exemple).
\end{rapport}

\secindispensables

\begin{itemize}
        \item Complexité en temps (meilleur cas, pire cas, moyenne)
        et en espace.
	\item Méthodes d'analyse
    \item Des exemples.
\end{itemize}

\secasavoir

\begin{itemize}
	\item Analyse des algorithmes : relations de comparaison $O$, $\theta$ et $\Omega$.
    \item Exemple d’analyse en moyenne : recherche d’un élément dans un tableau.
    \item Des algorithmes et leurs complexité dans différent domaine (minimisation d'automate, plus court chemins, unification, recherche... )
    	\item Complexité amortie par méthode de l'agrégat.

\end{itemize}

\secidees

\begin{itemize}
    	\item Complexité amortie par méthode comptable ou potentiel.
        \item FFT
        \item Compromis temps-mémoires, rainbow table.
\end{itemize}

\secpieges

\begin{itemize}
    \item Il ne faut pas se limiter à la complexité en temps, mais aussi parler de complexité en mémoire.
    \item Connaître la complexité des diverses opérations de base de chacun des algorithmes présentés.
    \item Il faut trouver un équilibre entre la présentation d'exemples, et la présentation formelle des notions théoriques.
    \item Si certains points sont présentés de manière informelle, ne pas les appeler proposition ou théorème.
    \item Faire le lien avec certaines méthodes et certains paradigmes
    de programmation (e.g, le théorème maître est à relier au
    paradigme diviser pour régner).
    \item Bien avoir en tête les mesures pour les tailles des entrées.
\end{itemize}

\secquestionsclassiques

\begin{itemize}
	\item Lors d'une analyse de complexité en moyenne, que
	suppose-t-on sur la distribution des entrées ? Est-ce pertinent ?
%     On suppose qu'elle est uniforme, c'est en réalité rarement le
%cas, mais c'est l'hypothèse la plus raisonnable.
    \item Pourquoi se contente-t-on de $O$, plutôt que de calculs
    exacts ?
%    On cherche à classifier les algorithmes par grande classe de
%complexité, stable par $O$. Les calculs exacts peuvent servir pour
%l'optimisation d'un problème précis.
    \item Pouvez vous dérouler tel algorithme sur tel exemple ?
    \item Questions de détails d'implémentation/de complexité des opérations élémentaire sur un algorithme présenté.
    \item Dans tel cas pratique, quelle algorithme vaut-il mieux utiliser ?
    \item Savez-vous si tel algorithme est optimal ?

\end{itemize}

\secreferences

\begin{itemize}
\item \input{refs/cormen}
\item \input{refs/beauquier}

\end{itemize}

\secdev

\begin{itemize}
\item[+] \input{dev/dijkstra}
\item[++] \input{dev/tri_rapide}
\item[++] \input{dev/Arbres2-4}
\item[++] \input{dev/hachageparfait}
\end{itemize}


\end{document}
%%% Local Variables:
%%% mode: latex
%%% TeX-master: t
%%% End:

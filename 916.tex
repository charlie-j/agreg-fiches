
\documentclass{agregfiche}

\title{Leçon 916 -- Formules du calcul propositionnel : Représentation, formes
normales, satisifiabilité. Applications.}

\begin{document}

\maketitle

\secrapports

\begin{rapport}{2017}
    Le jury attend des candidats qu’ils abordent les questions de la complexité de la satisfiabilité. Pour autant, les applications ne sauraient se réduire à la réduction de problèmes NP-complets à SAT. Une partie significative du plan doit être consacrée à la représentation des formules et à leurs formes normales.
\end{rapport}

\secindispensables

\begin{itemize}
    \item La syntaxe et la sémanitque des propositions,
        le lien avec les tables de vérité
    \item Faire la différence entre validité et satisfiabiltié dès le départ.
    \item Les méthodes syntaxiques de preuve (déduction naturelle,
        calcul des séquents, résolution)
    \item Les méthodes sémantiques de prevue (DPLL, tableau, 
        table de vérité, BDD)
    \item Les variantes de SAT et complexité
        (SAT, HORNSAT)
    \item Système de connecteurs, complétude (le XOR est complet).
    \item Formes normales (CNF, DNF, BDD),
        pour chacune la complexité des problèmes de décision, 
        et du passage d'une forme à l'autre (pour la validité ? pour la
        satisfiabilité ?)
\end{itemize}

\secidees

\begin{itemize}
    \item Le théorème de compacité est joli
    \item Justifier l'étude du fragment propositionnel 
        via Herbrand
    \item Évoquer la résolution au premier ordre
    \item Applications des SAT solvers 
        (HAMPATH, Sudoku, coloration de graphes). Attention les réductions sont \emph{à l'envers} !
    \item Affirmer plus de choses sur les BDD (représentation mémoire
        via hash-consing,
        étude de cas pathologiques).
    \item Passer plus de temps sur les systèmes complets, les problèmes sont-ils 
        plus ou moins durs en fonction du système de connecteur choisi~?
    \item Faire un peu de complexité en parlant de SAT, 3SAT, HORNSAT, et 2SAT
    \item Parler de la génération de circuits
\end{itemize}

\secpieges

\begin{itemize}
    \item Ne pas centrer toute la leçon sur $NP$
    \item Faire une grosse partie sur les formes normales même si cela semble
        vide
    \item Bien comparer les méthodes de résolution, leurs avantages et leurs
        inconvénients (vitesse, certificat)
\end{itemize}


\secquestionsclassiques

\begin{itemize}
    \item ??
\end{itemize}

\secreferences

\begin{itemize}
    \item \temporary{Goubault}
    \item \temporary{Cori Lascar 1/2}
    \item \temporary{Raffali}
    \item \temporary{René Lalement} 
\end{itemize}

\secdev

\begin{itemize}
    \item \temporary{Compacité du calcul propositionnel et application}
    \item \temporary{2SAT est $NL$-complet et en temps linéaire sur une 
        machine RAM}
    \item \temporary{Complétude de la résolution propositionnelle}
    \item \temporary{Théorème de Cook}
\end{itemize}


\end{document}

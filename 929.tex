\documentclass{agregfiche}

\title{Leçon 929 -- Lambda-calcul pur comme modèle de calcul. Exemples.}

\begin{document}

\maketitle

\secrapports

\begin{rapport}{2019}

Il s'agit de présenter un modèle de calcul : le lambda-calcul pur.
Il est important de faire le lien avec au moins un autre modèle de calcul, par exemple les machines de Turing ou les fonctions récursives.
Néanmoins, la leçon doit traiter des spécificités du lambda-calcul.
Ainsi, le candidat doit motiver l'intérêt du lambda-calcul pur sur les entiers et aborder la façon dont il permet de définir et d'utiliser des types de données (booléens, couples, listes, arbres).

\end{rapport}

\secindispensables

\begin{itemize}
    \item Définitions : lambda-termes, alpha-équivalence, substitution.
    \item $\beta$-redex, $\beta$-réduction, confluence.
   	\item Expressivité : équivalence avec les fonctions primitives.
\end{itemize}

\secasavoir

\begin{itemize}
	\item Représentation des données : entiers, booléens.
	\item Fonctions représentables.
  	\item Stratégies d'évaluation (gauche, tête).
	\item Exemples : fonctions arithmétiques, conditionnelle.
	\item Exemples : fonctions récursives avec combinateur de point
	fixe.
\end{itemize}

\secidees

\begin{itemize}
	\item Types de données : couples, tuples, listes, arbres, et leurs primitives.
	\item Développements finis.
	\item Fonctions fortement représentables.
	\item Lien avec les langages fonctionnels (appel par valeur, nom, need)
	\item Systèmes numériques (nombres de Gödel, Parigot, Church) et conversions
	\item Indécidabilité sur les $\lambda$-termes (cf Krivine/Barendregt)
\end{itemize}

\secpieges

\begin{itemize}
    \item Ne pas sortir du cadre "modèle de calcul".
    \item Surtout ne pas parler de système de type.
    \item Bien doser le temps passé sur la substution et l'alpha
    équivalence.
    \item Avoir en tête (ou dans ses notes) les petits exemples de
    bases (pour la non terminaison de la $\beta$-réduction, pour la
    différence entre fortement et faiblement normalisant,...)
    %   B = (lambda x.xx)(lambda x.xx), C= (lambda y.x)B,
    \item Voir le lien entre stratégie et appel par nom ou valeur (ou
    paresseuse)
    dans les langages de programmations (avec exemple concret de
    langages).
%    Ocaml et appel par nom, Haskell et appel par valeur
    \item Choisir un système numérique, et en connaitre au moins un
    autre.



\end{itemize}

\secquestionsclassiques

\begin{itemize}
    \item Quel est l'intérêt du $\lambda$-calcul (par rapport aux
    autres modèles de calcul)?
    \item Donner un contre exemple à la forte confluence.
%     (lambda x (x+x))(3+4)
    \item Connaissez vous une autre forme de confluence que la
    confluence ou la confluence forte ?
%     confluence local, qui implique la confluence forte quand on a
%une relation fortement normalisable
    \item Citer un combinateur de point fixe.
    \item Le combinateur de point fixe donné
        fonctionne-t-il correctement en appel
        par valeur ?
    \item Écrire la factorielle en $\lambda$-calcul
        dans le système numérique de votre choix.
    \item Calculer la forme normale d'un terme donné.
    \item Quel est l'intérêt de la représentation forte (par rapport à la faible)
    \item Est-ce qu'un terme est $\beta$-équivalent à une unique forme de tête ?
    \item Connaissez vous l'$\eta$-expansion ? À quoi peut-elle
    servir ?
    \item Connaissez-vous d'autres théorèmes de point fixe (même en
    dehors du $\lambda$-calcul) ?
    \item Peut-on décider si un terme a une forme normale ?
\end{itemize}

\secreferences

\begin{itemize}
    \item \input{refs/krivine}
\end{itemize}


\secdev

\begin{itemize}
    \item[++] \input{dev/lambdaequivrecursive}

\end{itemize}


\end{document}

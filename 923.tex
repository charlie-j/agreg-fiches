
\documentclass{agregfiche}

\title{Leçon 923 - Analyses lexicale et syntaxique. Applications.}

\begin{document}
\maketitle

\secrapports
\begin{rapport}{2018}
    Cette leçon ne doit pas être confondue avec la 909, qui s’intéresse aux seuls langages rationnels, ni avec
    la 907, sur l’algorithmique du texte.
    Si les notions d’automates finis et de langages rationnels et de grammaires algébriques sont au cœur
    de cette leçon, l’accent doit être mis sur leur utilisation comme outils pour les analyses lexicale et
    syntaxique. Il s’agit donc d’insister sur la différence entre langages rationnels et algébriques, sans perdre
    de vue l’aspect applicatif : on pensera bien sûr à la compilation. Le programme permet également des
    développements pour cette leçon avec une ouverture sur des aspects élémentaires d’analyse sémantique.
\end{rapport}

\begin{rapport}{2017}
[\dots]    On pourra s’intéresser à la transition entre analyse lexicale et analyse syntaxique, et on pourra présenter les outils associés classiques, sur un exemple
    simple. Les notions d’ambiguïté et l’aspect algorithmique doivent être développés. La présentation
    d’un type particulier de grammaire algébrique pour laquelle on sait décrire un algorithme d’analyse
    syntaxique efficace sera ainsi appréciée. [\dots]
    \end{rapport}

\secindispensables

\begin{itemize}
	\item Grammaires et Langages algébriques. Existence de langages non algébriques (Lemme d'\bsc{Ogden}). Propriétés de clôture
    des langages algébriques.
    \item Chaîne de compilation. Analyse lexicale. Analyse syntaxique (principes de l’analyse descendante
    et ascendante). 
\end{itemize}

\secasavoir

\begin{itemize}
    \item Formes normales. (quadratique, \bsc{Chomsky}, \bsc{Greibach})
	\item Automate à pile et équivalence avec les langages algébriques.
    \item Analyse sémantique \underline{élémentaire} (arbre de syntaxe abstraite, table des symboles,
    analyse de portée, typage, ...).
    \item Grammaires LL ou LR
    \item Ambiguïté.
\end{itemize}

\secidees

\begin{itemize}
	\item LALR ou SLR
    \item Problèmes de décision des langages algébriques. Décidabilité et complexité.
    \item Langages de pile.
    \item Hiérarchie de \bsc{Chomsky}.
\end{itemize}

\secpieges

\begin{itemize}
    \item Des DESSINS, de la chaîne de compilation, d'arbre de dérivation, d'automate, de règles de grammaires,...
    \item Mettre des exemples. Bien s'entraîner à faire tourner les algorithmes sur des exemples (notamment pour l'analyse lexical)
	\item Attention à ne pas trop se perdre sur certain théorique. On peut notamment se perdre et ne pas savoir répondre aux questions autour des parseurs LL, LR, LALR et SLR. 
    \item En pratique, yacc est LALR, mais par exemple, le langage pascal est LL(1)

\end{itemize}

\secquestionsclassiques

\begin{itemize}
	\item Qu'utilisent en pratique les programmeurs pour écrire des parseurs ? Quelle type de grammaire est reconnu par ces outils ?
% lex and yacc, LALR
\item Peut-on comparer les parseurs LALR, LR, LL, SLR ? Si  des inclusions sont strictes, avez-vous des exemples ?
% LALR < LR < SLR, LALR(j) incomparable avec LL(k)
\item Quelles sont les applications de l'analyse lexicale et 
syntaxique ? Avez-vous un exemple où la complexité est critique ?
% Compilations, notamment de page HTML (temps réel)
\item Complexité d'une mise en forme normale ?
% chomsky
\item Pourquoi séparer l'analyse en deux étapes ?
% Rend plus simple et efficace la conception de compilateurs. 
%L'analyse lexicale est la seule à manipuler le texte, à partir de la 
%syntaxique, on est indépendant des particularité du langage source.
\item Pourquoi l'ambiguïté d'une grammaire est problématique ?
% un programme doit avoir une unique interprétation

\end{itemize}

\secreferences

\begin{itemize}
\item \reference{Car}{Langages formels, calculabilité et complexité}{Carton}{à la BU/LSV}{Très bonne référence couvrant beaucoup de bases. Se méfier de certaines preuves faite un peu rapidement.}  

\item \reference{Aho}{Compilers: Principles, Techniques, and Tools}{\bsc{Aho} et. al}{à la BU/LSV}{La référence pour la compilation, surnommé le dragon. En réalité, seul deux/trois chapitres sont utiles.}  

\end{itemize}

\secdev

\begin{itemize}
\item[++] \dev{Algorithme de Cocke-Kasami-Younger}{[Hopcroft, Ullman]}{Ch7.4.4, p298}{923}{Uniquement esquissé dans le Carton. Maitriser la mise en forme normale de Chomsky.}

\item[++] \dev{Un exemple d'analyse d'un langage jouet}{[Aho]}{Ch4.4 ou Ch 4.9.1 p298}{923}{Le langage d'une calculatrice minimaliste est un classique. L'analyse d'une grammaire LL(1) est plus difficle. Ne pas hésiter à faire des dessins, avec des arbres d'interprétation possible.}

\item[-] \dev{Problèmes indécidables pour les grammaires algébriques.}{[Car]}{}{914,923}{Ambigüité, universalité.}

\item[-] \dev{Théorème de Klenne}{IDref}{p. 432}{lecon1,lecon2,lecon3}{Commentaires}

\end{itemize}


\end{document}
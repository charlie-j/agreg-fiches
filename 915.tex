\documentclass{agregfiche}

\title{Leçon 915 -- Classes de complexité. Exemples}

\begin{document}

\maketitle

\secrapports

\begin{rapport}{2018}

Le jury attend que le candidat aborde à la fois la complexité en temps et en
espace. Il faut naturellement exhiber des exemples de problèmes appartenant aux
classes de complexité introduites, et montrer les relations d’inclusion
existantes entre ces classes, en abordant le caractère strict ou non de ces
inclusions. Le jury s’attend à ce que les notions de réduction polynomiale, de
problème complet pour une classe, de robustesse d’une classe vis à vis des
modèles de calcul soient abordées. Parler de décidabilité dans cette leçon
serait hors sujet.

\end{rapport}

\secindispensables

\begin{itemize}
    \item Complexité en espace et en temps.
    \item Les classes usuelles ($\P$, $\NP$, $\PSPACE$) et inclusions.
    \item Notion de réduction, de problème complet
    
%    \item Le problème "classiquement complet"
%        \begin{equation}
%            L = \left\{ (M, x, 1^t) ~|~ \langle M, x \rangle \text{ termine en
%                temps (resp. espace) }\leq t
%            \right\}
%        \end{equation}

\end{itemize}

\secasavoir

\begin{itemize}

	\item La robustesse vis-à-vis du modèle de calcul 
	(déterministe, non-déterministe, une bande, plusieurs 
	bandes, écriture sur entrée)
	\item Savitch et les classes en espace
	\item Exemples de problèmes complets sur des 
	domaines différents (graphes, automates,
	grammaires, logique, circuits)
	\item Les théorèmes de hiérarchie stricte
\end{itemize}


\secidees

\begin{itemize}
    \item Les classes $NL$, $co-NL$ et Immerman-Szelepcsényi
    \item Développer le côté logique (2SAT, HORNSAT, SAT, QBF, CTLSAT).
    \item Des théorèmes de borne inférieure de simulation
(une bande vs deux bandes en $O(n)$ vs $O(n^2)$ 
pour le langage des palindromes).
    \item Algorithmes randomisés comme autre modèle de calcul
        (tests de primalité, algorithme de Berlekamp,
        polynomial identity testing, RP, coRP)
    \item Les classes alternantes, le lien avec 
        EXPSACE et la hiérarchie polynômiale (Carton)   
    \item Une machine qui calcule trop rapidement est un automate 
        (Carton)
\end{itemize}

\secpieges

\begin{itemize}
    \item Ne pas centrer toute la leçon sur $P$ et $NP$
  	\item Un petit diagramme qui permet de s'y retrouver 
    en annexe avec les inclusions strictes et les égalités est le bienvenue.
    \item Si on parle de $NL$, bien formaliser le type de machines 
        considérées,  et bien expliquer la notion de réduction polynômiale,
        et logarithmique si on parle de $NL$
    \item Les différentes ouvertures hors programme peuvent être très dangereuse, le domaine de la complexité est plein de pièges et de suptilité.

\end{itemize}


\secquestionsclassiques

\begin{itemize}
    \item Quels sont les impacts de la variation du modèle sur la complexité ?
    \item Pourquoi utiliser le formalisme des machines de Turing 
        plutôt qu'un autre ($\lambda$-calcul, fonctions récursives) ?
    \item Citer une classe qui n'est pas $P$, $NP$ ou $PSPACE$.
    \item Montrer que tel problème $X$ est $C$-complet.
    \item Comment varie la complexité en fonction de la taille de l'alphabet ?
\end{itemize}

\secreferences

\begin{itemize}
\item \reference{Car}{Langages formels, calculabilité et complexité}{Carton}{à la BU/LSV}{Très bonne référence couvrant beaucoup de bases. Se méfier de certaines preuves faite un peu rapidement.}  

\item 
\reference{Aro}{Computational Complexity: A Modern Approach}{Arora \bsc{Barak}}{au LSV}{En anglais. Les premiers chapitres couvrent les notions nécéssaires et donnent souvent une bonne intuition. Manque parfois de formalisme, à croiser avec le Perifel.}  

\end{itemize}

\secdev

\begin{itemize}
    \item \dev{Théorème de Cook-Levin}{[Car]}{p. 191}{915,916,928}{Preuve que 3-SAT est NP-complet.}


%    \item \temporary{Hiérarchie en temps et en espace}
%    \item \temporary{Immerman Szelepcseniy}
%    \item \temporary{Universalité d'un langage rationnel est PSPACE complet}
\end{itemize}


\end{document}

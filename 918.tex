
\documentclass{agregfiche}

\title{Leçon 918 - Systèmes formels de preuve en logique du premier ordre. Exemples.}

\begin{document}
\maketitle

\secrapports
\begin{rapport}{2018}
	Le jury attend du candidat qu’il présente au moins la déduction naturelle ou un calcul de séquents
	et qu’il soit capable de développer des preuves dans ce système sur des exemples classiques simples.
	La présentation des liens entre syntaxe et sémantique, en développant en particulier les questions de
	correction et complétude, et de l’apport des systèmes de preuves pour l’automatisation des preuves est
	également attendue.
	Le jury appréciera naturellement si des candidats présentent des notions plus élaborées comme la
	stratégie d’élimination des coupures mais est bien conscient que la maîtrise de leurs subtilités va au-
	delà du programme.
\end{rapport}


systèmes formels de preuve. 
Algorithme d’unification des termes. Preuves par résolution.

\secindispensables

\begin{itemize}
	\item Un syst1eme de preuve, et des preuves.
	\item Syntaxe et Sémantique, correction et complétudes.
\end{itemize}

\secasavoir

\begin{itemize}
	\item Calcul des séquents, déduction naturelle.
	\item Algorithme d'unification.
	\item Preuves par résolution.
	\item Automatisation des preuves
\end{itemize}

\secidees

\begin{itemize}
	\item Élimination des coupures
	\item Logique intuitionniste.
	\item Compacité
	\item Méthodes des tableaux
	\item Théories équationelle et réécriture
	\item Clause de Horn, résolution et Prolog
\end{itemize}

\secpieges

\begin{itemize}
	\item Éviter de mettre les règles en annexes. Des abres de preuves peuvent y trouver leurs places
	\item Ne pas mettre tous les systèmes. Attention à bien justifier l'introduction des différents systèmes, on ne fait pas un catalogue.
	\item Cette leçon ne porte pas sur les théories.
\end{itemize}

\secquestionsclassiques

\begin{itemize}
	\item Discuter des difficultés de l'automatisation dans différents systèmes.
	\item Démontrer tel formule dans tel système.
	\item La logique du premier ordre est-elle décidable ?
	\item Motivation dérrière le développements des systèmes à la Hilbert ?
	\item Connaissez vous des outils (programmes) permettant de travailler avec un système de preuve et faire des preuves formelles ?
\end{itemize}

\secreferences

\begin{itemize}
\item \reference{IDref}{Nom du bouquin}{Auteur1,Auteur2}{à la BU/LSV}{Des commentaires trop ouf qui déchirent}  
\item \temporary{Goubault}
\item \temporary{Dowel}
\end{itemize}

\secdev

\begin{itemize}
\item \temporary{Théorème de Herbrand (Goubault)}
\item \temporary{Algorithme naif d'unification (DAVID)}
\item \temporary{élimination des coupures (dowek?)}
\item \temporary{complétude de la résolution (Stern?)}
\end{itemize}


\end{document}
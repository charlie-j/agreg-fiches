
\documentclass{agregfiche}

\title{Leçon 924 - Théories et modèles en logique du premier ordre. 
Exemples.}

\begin{document}
\maketitle

\secrapports
\begin{rapport}{2018}
    Le jury s’attend à ce que la leçon soit abordée dans l’esprit de 
    l’option informatique, en insistant plus
    sur la décidabilité/indécidabilité des théories du premier ordre 
    que sur la théorie des modèles.
    Il est attendu que le candidat donne au moins un exemple de 
    théorie décidable (respectivement com-
    plète) et un exemple de théorie indécidable. Si le jury peut 
    s’attendre à ce que le candidat connaisse
    l’existence du théorème d’incomplétude, il ne s’attend pas à ce 
    que le candidat en maîtrise la démons-
    tration.
\end{rapport}

\secindispensables

\begin{itemize}
	\item Syntax et Sémantique. Interprétation d’une formule dans un 
	modèle. Validité et satisfiabilité. 
    \item Cohérence et complétude. 
    \item Décidabilité et indécidabilité
    \item Exemples de théories.
\end{itemize}





\secasavoir

\begin{itemize}
	\item Théorie de l'égalité, arithmétique de \bsc{Peano}.
    \item  Théorème de complétude du calcul des prédicats du premier 
    ordre.
    \item Théorème d'incomplétude.
    \item Théorème de compacité.
\end{itemize}

\secidees

\begin{itemize}
    \item Théorie de \bsc{Presburger}, Groupes.
	\item Théorème de \bsc{Lowemheim}-\bsc{Skolem}.
    \item Modèle et Théorème de \bsc{Herbrand}, complétude de la 
    résolution.
    \item Axiome indépendant.
    \item Élimination des quantificateurs.
\end{itemize}

\secpieges

\begin{itemize}
	\item Sujet vaste où il est facile de se perdre. Il faut essayer 
	de bien comprendre la signification et les implications des 
	théorèmes et des propriétés, et non juste leur formulation.
    \item Il faut avoir en tête (ou dans ses notes) des petits 
    exemples simples de théories de différents types.
    \item Avoir un petit exemple de preuve dans une certaine théorie 
    (par exemple l'égalité).
    \item On peut montrer la décidabilité de l'arithmétique de 
    \bsc{Presburger} avec des automates. On peut montrer 
    l'indécidabilité 
    de l'arithmétique de \bsc{Peano} en passant par l'indécidabilité 
    de la 
    terminaison d'un programme. On peut considérer l'ensemble des 
    relations définissables, ce qui mène a considérer les bases de 
    données relationnelles. Les modèles de \bsc{Herbrand} mène à la 
    complétude de la méthode de résolution. Bref, il y a des liens 
    avec les autres 
    leçons, il faut en avoir conscience.
    \item Ne pas confondre la notion de complétude d'un système de 
    preuve et la complétude d'une théorie.
    \item Il peut être bien d'avoir quelques idées d'applications en 
    tête : il existe des groupes de toute cardinalités, ...
\end{itemize}

\secquestionsclassiques

\begin{itemize}
	\item Est-ce que la théorie des groupes est cohérente ? complètes 
	?
%    Oui, car il existe un groupe. Non, on exhibe une formule vraie 
%dans un modèle et pas dans un 
%autre, par ex l'idempotence.
    \item Exemple d'un modèle non standard de l'arithmétique de 
    \bsc{Peano}
    ?
    \item Comment prouver qu'une théorie n'est pas contradictoire ?
%    Par le théorème de correction, il suffit d'exhiber un modèle 
%satisfiant ses axiomes. En effet, un modèle ne peut satisfaire à la 
%fois un thm et sa négation.
    \item Un exemple de théorie complète ? décidable ? indécidable ? 
    ...
    \item Peut-on avoir une théorie complète et décidable ? complète 
    et indécidable ? incomplète et décidable ?
%    Prendre n'importe qu'elle th complète, car Complète => 
%décidable. non. Oui, l'égalité sans symbole de function, avec F = 
%forall x y. x=y
    \item La théorie vide est-elle décidable ? Complète ?
% indécidable, par le théorème de church turing. indécidable => 
%incomplétude
    \item  Est-ce que $T \vdash A \Leftrightarrow B$ si et seulement 
    si $T 
    \vdash A \Leftrightarrow T\vdash B$ ?
%  Non,  T = emptyset, A = p(), B = q()
\end{itemize}

\secreferences

\begin{itemize}
\item \reference{Go}{Proof Theory and Automated Deduction}{Jean Goubault-Larrecq}{à la BU/LSV}{Le must pour la logique}  

\item \reference{Da}{Introduction à la logique}{R. \bsc{David}, K. \bsc{Nour}, C. \bsc{Raffalli}}{à la BU/LSV}{L'autre must pour la logique}  

\item \reference{Dow}{Les démonstrations et les algorithmes - Introduction 
à la logique et à la calculabilité}{Gilles \bsc{Dowek}}{à la BU 
?}{Très bien pour prendre du recul.}  

\end{itemize}

\secdev

\begin{itemize}
%\item \dev{Elimination des quantificateurs dans la théorie des ordres linéaires}{?}{}{914,924}{}

\item \dev{Indécidabilité de l'arithmétique de 
\bsc{Peano}}{?}{}{914,924}{Passer par l'encodage des fonctions 
calculables. Assez long si on fait tout.}

\item \dev{Décidabilité de l'arithmétique de \bsc{Presburger}}{[Car]}{Thm 3.63 p.164}{909,914,924}{Idée générale simple, mais attention aux détails. Réfléchir au codage, à sa sémantique, et à la complexité globale de la construction.}

\end{itemize}


\end{document}
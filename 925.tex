\documentclass{agregfiche}

\title{Leçon 925 -- Graphes : représentations et algorithmes.} 

\begin{document}

\maketitle

\secrapports

\begin{rapport}{2017}

Cette leçon offre une grande liberté de choix au candidat, qui peut décider de présenter des algorithmes sur des problèmes variés : connexité, diamètre, arbre couvrant, flot maximal, plus court chemin, cycle eulérien, etc. mais aussi des problèmes plus difficiles, comme la couverture de sommets ou la recherche d’un cycle hamiltonien, pour lesquels il pourra proposer des algorithmes d’approximation ou des heuristiques usuelles. Une preuve de correction des algorithmes proposés sera évidemment appréciée. Il est attendu que diverses représentations des graphes soient présentées et comparées, en particulier en termes de complexité.

\end{rapport}

\secindispensables

\begin{itemize}
    \item Les représentations des graphes et leurs complexités
        (bien insister dessus à chaque algorithme)
    \item Dijkstra

\end{itemize}

\secasavoir
\begin{itemize}
	\item Les algorithmes de parcours de graphe
	\item Les algorithmes de plus courts chemins
\end{itemize}
\secidees

\begin{itemize}
	\item Algorithmes de composantes connexes (Kozaraju, Tarjan)
	\item Arbres couvrants
	\item Problèmes de flots
\end{itemize}

\secpieges

\begin{itemize}
    \item Ne pas essayer de faire un plan trop original. Cette leçon est une
        liste d'algorithmes !
    \item Donner des applications aux algorithmes (approximations de problèmes
        NP, etc ...)
    \item Ne pas faire une partie "problèmes NP", ou plus généralement 
        sur la théorie des graphes
\end{itemize}

\secquestionsclassiques

\begin{itemize}
    \item TODO
\end{itemize}

\secreferences

\begin{itemize}
    \item Cormen
    \item Beauquier
    \item Dalpagusta
\end{itemize}

\secdev

\begin{itemize}
    \item Correction totale et complexité de Dijkstra (Cormen / Beauquier)
    \item 2SAT est NL-complet et se résout en temps polynomial (Carton)
\end{itemize}

\end{document}

\documentclass{agregfiche}

% v1 par Alliaume
% Relu par Charlie Jacomme
% Relu par Thomas Chatin


\title{Leçon 925 -- Graphes : représentations et algorithmes.}

\begin{document}

\maketitle

\secrapports

\begin{rapport}{2017,2018,2019}

Cette leçon offre une grande liberté de choix au candidat, qui peut décider de présenter des algorithmes sur des problèmes variés : connexité, diamètre, arbre couvrant, flot maximal, plus court chemin, cycle eulérien, etc. mais aussi des problèmes plus difficiles, comme la couverture de sommets ou la recherche d’un cycle hamiltonien, pour lesquels il pourra proposer des algorithmes d’approximation ou des heuristiques usuelles. Une preuve de correction des algorithmes proposés sera évidemment appréciée. Il est attendu que diverses représentations des graphes soient présentées et comparées, en particulier en termes de complexité.

\end{rapport}

\secindispensables

\begin{itemize}
    \item Représentations des graphes, complexités des opérations élémentaires.
    \item Parcours en largeur et en profondeur.
    \item Plus courts chemins, Dijkstra.
\end{itemize}

\secasavoir
\begin{itemize}
    \item Plus courts chemins, Floyd Warshall, Bellman Ford.
    \item Arbres couvrants, Prim, Kruskal.
    \item Fermeture transitive d'une relation. (notamment par adaptation de Floyd Warshall)
    \end{itemize}
\secidees

\begin{itemize}
    \item Algorithmes de composantes connexes (Kozaraju, Tarjan)
    \item Cycle eulérien, hamiltoniens.
    \item Problèmes de flots.
    \item Lien avec la complexité et la $NP$-complétude.
\end{itemize}

\secpieges

\begin{itemize}
    \item Faire des dessins.
    \item Avoir écrit des algos dans le plan peux permettre d'éviter des les réécrire pendant un développement.
    \item Ne pas essayer de faire un plan trop original. Cette leçon est une liste d'algorithmes ! Il faut cependant essayer de justifier l'ordre de présentation, et mettre du liant.
    \item Donner des applications aux algorithmes (approximations de problèmes NP, etc ...)
    \item Ne pas faire une partie "problèmes NP", ou plus généralement sur la théorie des graphes.
    \item Ne pas se restreindre aux graphes orientés.
    \item Pour les graphes pondérés, ne pas prendre des poids réels, uniquement des entiers.
\end{itemize}

\secquestionsclassiques

\begin{itemize}
    \item Exemple d'application de la détection de cycle, de flots, de ... ?
    \item Pourquoi les graphes sont-ils si importants ?
    \item Dérouler tel algorithme sur tel exemple.
    \item Qu'est-ce qui peut influencer le choix entre les différents algorithmes de plus courts chemins.
    \item Est-ce que le choix de la représentation influe fortement sur la complexité en espace des algorithmes ?
    \item Donner un exemple d'application des tris topologiques ?
      % Gestion des dépendances dans le programme Make par exemple.
\end{itemize}

\secreferences

\begin{itemize}
    \item \input{refs/cormen}
    \item \input{refs/beauquier}
%    \item Dalpagusta
\end{itemize}

\secdev

\begin{itemize}
	\item[++] \input{dev/dijkstra}
	\item[++] \input{dev/primkruskal}
        \item[-]\input{dev/2SAT}
          \textit{Bien insister sur les algorithmes de graphes utilisés (Kosaraju par exemple), et bien justifier que le temps reste linéaire en la taille de la formule.}
\end{itemize}

\end{document}
